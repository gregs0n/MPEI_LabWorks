\documentclass[a4paper,12pt]{report} % добавить leqno в [] для нумерации слева

%%% Работа с русским языком
\usepackage{cmap}					% поиск в PDF
\usepackage{mathtext} 				% русские буквы в формулах
\usepackage[T2A]{fontenc}			% кодировка
\usepackage[utf8]{inputenc}			% кодировка исходного текста
\usepackage[english,russian]{babel}	% локализация и переносы

%%% Дополнительная работа с математикой
\usepackage{amsmath,amsfonts,amssymb,amsthm,mathtools} % AMS
\usepackage{icomma} % "Умная" запятая: $0,2$ --- число, $0, 2$ --- перечисление

%% Номера формул
\mathtoolsset{showonlyrefs=true} % Показывать номера только у тех формул, на которые есть \eqref{} в тексте.

%% Шрифты
\usepackage{euscript}	 % Шрифт Евклид
\usepackage{mathrsfs} % Красивый матшрифт

%% Свои команды
\DeclareMathOperator{\sgn}{\mathop{sgn}}

\setlength\parindent{0ex}
\setlength\parskip{0.3cm}

%%% Заголовок
\author{Волков Павел А-14-19}
\title{Типовой расчет №1 по численным методам Вариант 3}
\date{\today}

\begin{document} % конец преамбулы, начало документа

\maketitle

\newpage

\section*{Задание}
Вычислить значение $Z$ и оценить абсолютную и относительную
погрешноcти результата, считая, что значения исходных данных получены в результате округления по дополнению.
Записать результат с учетом погрешности. Указать верные цифры.
\[
    Z = 5.05^2 - 0.21 - \frac{1}{1.718}
\]

\section*{Решение}
Вычислим значение: $Z = 24.710427823$

Для оценки погрешности введем функцию:
$Z(x, y, k) = x^2 - y + \frac{1}{k}$

\begin{gather*}
    x^* = 5.05 \pm 0.005\\
    y^* = 0.21 \pm 0.005\\
    k^* = 1.718 \pm 0.0005
\end{gather*}

По формуле оценки погрешности функции многих переменных имеем:

$\partial Z / \partial x = 2x$
\newline$\partial Z / \partial x |_{x=5.05} = 2 \cdot 5.05 = 10.1$

$\partial Z / \partial y = -1 $

$\partial Z / \partial k = -\frac{1}{k^2} $
\newline$\partial Z / \partial k |_{k=1.718} = -\frac{1}{1.718^2} = 0.338808019 $

$\Delta Z = 10.1 \cdot 0.005 + 1 \cdot 0.005 + 0.338808019 \cdot 0.0005 = 0.055669404$

Величину погрешности округляем до 2-х значащих цифр: $\Delta Z=0.056$

В числе $Z$ получили 3 верные цифры. Найдем относительную погрешность:
\newline$\delta Z = 0.056 / 24.710427823 = 0.0023$

Ответ: $ Z = 24.710 \pm 0.056$, верных цифр - 3, $\delta Z = 0.2\%$

\end{document} % конец документа