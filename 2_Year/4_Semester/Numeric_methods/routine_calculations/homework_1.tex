\documentclass[a4paper,12pt]{report} % добавить leqno в [] для нумерации слева

%%% Работа с русским языком
\usepackage{cmap}					% поиск в PDF
\usepackage{mathtext} 				% русские буквы в формулах
\usepackage[T2A]{fontenc}			% кодировка
\usepackage[utf8]{inputenc}			% кодировка исходного текста
\usepackage[english,russian]{babel}	% локализация и переносы

\usepackage{graphicx}				%вставка изображений(графиков, в частности)

%%% Дополнительная работа с математикой
\usepackage{amsmath,amsfonts,amssymb,amsthm,mathtools} % AMS
\usepackage{icomma} % "Умная" запятая: $0,2$ --- число, $0, 2$ --- перечисление

%% Номера формул
\mathtoolsset{showonlyrefs=true} % Показывать номера только у тех формул, на которые есть \eqref{} в тексте.

%% Шрифты
\usepackage{euscript}	 % Шрифт Евклид
\usepackage{mathrsfs} % Красивый матшрифт

%% Свои команды
\DeclareMathOperator{\sgn}{\mathop{sgn}}

%\setlength\parindent{0ex}
%\setlength\parskip{0.3cm}

%%% Заголовок
\author{Волков Павел А-14-19}
\title{ Домашнее задание по численным методам}
\date{\today}

\begin{document}

\section*{Задача 1}

Пусть $[ 0, 0.5]$ - отрезок локализации корня. Сколько шагов метода бисекции нужно сделать для нахождения корня с точностью $\varepsilon = 10^{-12}$.

\subsection*{Решение}
Число итераций метода бисекции для нахождения корня с точностью $10^{-k}$ на отрезке локализации длиной 1 можно расчитать по формуле:
\[
	N = \lceil \log_{2^{-1}}10^{-k}\rceil = \lceil k\cdot \log_2 10\rceil
\]
Так как на каждой итерации длина отрезка, а следовательно и погрешность решения уменьшается в 2 раза, то за $n$ итераций, она уменьшится в $2^n$ раз. Получаем уравнение $2^{-n} = 10^{-k}$.

Так как в условии задачи указан отрезок локализации $[0, 0.5]$, то можно считать, что одна итерация уже выполнена. Таким образом, из результата выше необходимо вычесть еще единицу.

Получаем ответ:$ N =  \lceil 12\cdot \log_2 10\rceil - 1 = \lceil 12 \cdot 3.321928 \rceil - 1 = 39$

\section*{Задача 2}
Запишите расчетные формулы модифицированного метода Ньютона для решения уравнения $f(x) = (x - \sin x)^2 = 0$. Предварительно оцените кратность корня.

\subsection*{Решение}

Ясно, что выражение в скобках имеет только один корень $x = 0$ и, так как оно стоит под квадратом, то данный корень имеет кратность 2.

Вычислим производную функции: $f'(x) = 2(x - \sin x) (1 - \cos x)$
\[
	x^{(k+1)} = x^{(k)} - 2 \frac{(x - \sin x)^2}{2(x - \sin x) (1 - \cos x)} = x^{(k)} - \frac{x - \sin x}{1 - \cos x}
\]

Итоговая расчетная формула: $x^{(k+1)} = x^{(k)} - \dfrac{x - \sin x}{1 - \cos x}$

\end{document}



