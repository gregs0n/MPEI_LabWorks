\documentclass[a4paper,12pt]{report} % добавить leqno в [] для нумерации слева

%%% Работа с русским языком
\usepackage{cmap}					% поиск в PDF
\usepackage{mathtext} 				% русские буквы в формулах
\usepackage[T2A]{fontenc}			% кодировка
\usepackage[utf8]{inputenc}			% кодировка исходного текста
\usepackage[english,russian]{babel}	% локализация и переносы

%%% Дополнительная работа с математикой
\usepackage{amsmath,amsfonts,amssymb,amsthm,mathtools} % AMS
\usepackage{icomma} % "Умная" запятая: $0,2$ --- число, $0, 2$ --- перечисление

%% Номера формул
\mathtoolsset{showonlyrefs=true} % Показывать номера только у тех формул, на которые есть \eqref{} в тексте.

%% Шрифты
\usepackage{euscript}	 % Шрифт Евклид
\usepackage{mathrsfs} % Красивый матшрифт

%% Свои команды
\DeclareMathOperator{\sgn}{\mathop{sgn}}

%\setlength\parindent{0ex}
%\setlength\parskip{0.3cm}

%%% Заголовок
\author{Волков Павел А-14-19}
\title{Типовой расчет №8 по численным методам Вариант 3}
\date{\today}

\usepackage{graphicx}

\begin{document} % конец преамбулы, начало документа

\maketitle

\newpage
\section*{Задание}
Решить систему уравнений $Ax = b$ методом прогонки

УКАЗАНИЕ. Промежуточные результаты вычислять с шестью знаками после запятой.
\[
	A = 
	\begin{pmatrix}
		8 & 4 & 0 & 0 & 0 \\
		-1 & 12 & -5 & 0 & 0 \\
		0 & 2 & 8 & 3 & 0 \\
		0 & 0 & -2 & 6 & 2 \\
		0 & 0 & 0 & 4 & 7
	\end{pmatrix}, b = 
	\begin {pmatrix}
		12 \\ 11 \\ -98 \\ -2 \\ -9
	\end{pmatrix}
\]

\section*{Решение}
Вычислим прогоночные коэффициенты:

Прямой ход прогонки:
\begin{gather*}
	\gamma_1 = 8.000000, \alpha_1 = -0.500000, \beta_1 = 1.500000 \\
	\gamma_2 = 12.500000, \alpha_2 = 0.400000, \beta_2 = 1.000000 \\
	\gamma_3 = 8.800000, \alpha_3 = -0.340909, \beta_3 = -11.136364 \\
	\gamma_4 = 6.681818, \alpha_4 = -0.299320, \beta_4 = -3.700680 \\
	\gamma_5 = 5.802721, \alpha_5 = 0.000000, \beta_5 = 1.000000 \\
\end{gather*}

Второй этап - обратный ход:

\begin{gather*}
	x_5 = \beta_5 = 1.000000 \\
	x_4 = \alpha_4 x_5 + \beta_4 = -0.299320 \cdot (1.000000) -3.700680 = -4.000000 \\
	x_3 = \alpha_3 x_4 + \beta_3 = -0.340909 \cdot (-4.000000) - 11.136364 = -10.000000 \\
	x_2 = \alpha_2 x_3 + \beta_2 = 0.400000 \cdot (-10.000000) + 1.000000 = -3.000000 \\
	x_1 = \alpha_1 x_2 + \beta_1 = -0.500000 \cdot (-3.000000) + 1.500000 = 3.000000 \\
\end{gather*}

Ответ: $(3, -3, -10, -4, 1)^T$

\end{document}